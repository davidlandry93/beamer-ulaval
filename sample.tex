\documentclass{beamer}

\usepackage[utf8]{inputenc}
\usepackage[T1]{fontenc}
\usepackage{graphicx}
\usepackage{booktabs}

\usetheme{ulaval}

\title[Titre court]{Titre complet plus long} 

\author{Jean Héacé} 
\institute[ulaval] 
{
Université Laval \\ 4
\medskip
\textit{jean.heace.1@ulaval.ca}
}
\date{\today}

\begin{document}

\begin{frame}
\titlepage
\end{frame}

\begin{frame}
\frametitle{Aperçu}
\tableofcontents 
\end{frame}


\section{Première section}

\subsection{Exemple de sous-section}

\begin{frame}
\frametitle{Paragraphes de tests}
Sed iaculis dapibus gravida. Morbi sed tortor erat, nec interdum arcu. Sed id lorem lectus. Quisque viverra augue id sem ornare non aliquam nibh tristique. Aenean in ligula nisl. Nulla sed tellus ipsum. Donec vestibulum ligula non lorem vulputate fermentum accumsan neque mollis.\\~\\

Sed diam enim, sagittis nec condimentum sit amet, ullamcorper sit amet libero. Aliquam vel dui orci, a porta odio. Nullam id suscipit ipsum. Aenean lobortis commodo sem, ut commodo leo gravida vitae. Pellentesque vehicula ante iaculis arcu pretium rutrum eget sit amet purus. Integer ornare nulla quis neque ultrices lobortis. Vestibulum ultrices tincidunt libero, quis commodo erat ullamcorper id.
\end{frame}

\begin{frame}
\frametitle{Liste}
\begin{itemize}
\item Lorem ipsum dolor sit amet, consectetur adipiscing elit
\item Aliquam blandit faucibus nisi, sit amet dapibus enim tempus eu
\item Nulla commodo, erat quis gravida posuere, elit lacus lobortis est, quis porttitor odio mauris at libero
\item Nam cursus est eget velit posuere pellentesque
\item Vestibulum faucibus velit a augue condimentum quis convallis nulla gravida
\end{itemize}
\end{frame}

\begin{frame}
\frametitle{Blocs de texte}
\begin{block}{Bloc 1}
Lorem ipsum dolor sit amet, consectetur adipiscing elit. Integer lectus nisl, ultricies in feugiat rutrum, porttitor sit amet augue. Aliquam ut tortor mauris. Sed volutpat ante purus, quis accumsan dolor.
\end{block}

\begin{block}{Bloc 2}
Pellentesque sed tellus purus. Class aptent taciti sociosqu ad litora torquent per conubia nostra, per inceptos himenaeos. Vestibulum quis magna at risus dictum tempor eu vitae velit.
\end{block}

\begin{block}{Bloc 3}
Suspendisse tincidunt sagittis gravida. Curabitur condimentum, enim sed venenatis rutrum, ipsum neque consectetur orci, sed blandit justo nisi ac lacus.
\end{block}
\end{frame}

\begin{frame}
\frametitle{Multiples Colonnes}
\begin{columns}[c] % The "c" option specifies centered vertical alignment while the "t" option is used for top vertical alignment

\column{.45\textwidth} % Left column and width
\textbf{En-tête}
\begin{enumerate}
\item Affirmation
\item Explication
\item Exemple
\end{enumerate}

\column{.5\textwidth} % Right column and width
Lorem ipsum dolor sit amet, consectetur adipiscing elit. Integer lectus nisl, ultricies in feugiat rutrum, porttitor sit amet augue. Aliquam ut tortor mauris. Sed volutpat ante purus, quis accumsan dolor.

\end{columns}
\end{frame}

\section{Seconde section}

\begin{frame}
\frametitle{Tableau}
\begin{table}
\begin{tabular}{l l l}
\toprule
\textbf{Traitements} & \textbf{Réponse 1} & \textbf{Réponse 2}\\
\midrule
Traitement 1 & 0.0003262 & 0.562 \\
Traitement 2 & 0.0015681 & 0.910 \\
Traitement 3 & 0.0009271 & 0.296 \\
\bottomrule
\end{tabular}
\caption{Légende}
\end{table}
\end{frame}

%------------------------------------------------

\begin{frame}
\frametitle{Théorème}
\begin{theorem}[Équivalence masse-énergie]
$E = mc^2$
\end{theorem}
\end{frame}


\begin{frame}[fragile]
\frametitle{Verbatim}
\begin{example}[Code de la diapo précédente] % Semble mal traduit par babel!
\begin{verbatim}
\begin{frame}
\frametitle{Théorème}
\begin{theorem}[Équivalence masse-énergie]
$E = mc^2$
\end{theorem}
\end{frame}\end{verbatim}
\end{example}
\end{frame}

%------------------------------------------------

\begin{frame}
\frametitle{Figure}
\begin{figure}
\includegraphics[scale=0.3]{ulaval.png}
\caption{Légende}
\end{figure}
\end{frame}

%------------------------------------------------

\begin{frame}[fragile] 
\frametitle{Citation}
Un exemple de la commande \verb|\cite| pour citer pendant la présentation:\\~

Référence nécessaire \cite{p1}.
\end{frame}

%------------------------------------------------

\begin{frame}
\frametitle{Références}
\footnotesize{
\begin{thebibliography}{99} 
\bibitem[Smith, 2012]{p1} John Smith (2012)
\newblock Titre de la publication
\newblock \emph{Nom du journal} 12(3), 45 -- 678.
\end{thebibliography}
}
\end{frame}

%------------------------------------------------

\begin{frame}
\Huge{\centerline{La fin}}
\end{frame}

%----------------------------------------------------------------------------------------

\section{Trouble making section}

\begin{frame}
    test
\end{frame}

\end{document}
